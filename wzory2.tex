\documentclass{article}
\usepackage[a4paper,left=3.5cm,right=2.5cm,top=2.5cm,bottom=2.5cm]{geometry}
\usepackage[MeX]{polski}
\usepackage[cp1250]{inputenc}
%%\usepackage{polski}
%%\usepackage[utf8]{inputenc}
\usepackage[pdftex]{hyperref}
\usepackage{makeidx}
\usepackage[tableposition=top]{caption}
\usepackage{algorithmic}
\usepackage{graphicx}
\usepackage{enumerate}
\usepackage{multirow}
\usepackage{amsmath} %pakiet matematyczny
\usepackage{amssymb} %pakiet dodatkowych symboli
\begin{document}
Tu umieszczamy kod TeXa, ktory bedzie kompilowany, $a^2$ a suma $\sum_{i=0}^{\infty}{2^i}$
\begin{displaymath}
\sum_{i=0}^{\infty}{2^i}
\end{displaymath}
\begin{equation}
\label{eq:iloczyn}
\prod^{n=i^2}_{i=2}=\frac{\lim^{n\rightarrow4}(1+\frac{1}{n})^n}{\sum k (\frac{1}{n})}
\end{equation}
\begin{displaymath}
\int^{\infty}_{2} {\frac{1}{log_{2}x}}dx=\frac{1}{x}sinx=1-cos^2(x)
\end{displaymath}
Na podstawie r�wnania \ref{eq:iloczyn}
\begin{equation}
\left[
\begin{array}{cccc}
a_{11} & a_{12} & \cdots & a_{1K} \\
a_{21} & a_{22} & \cdots & a_{2K} \\
\vdots & \vdots & \ddots & \vdots \\
a_{K1} & a_{K2} & \cdots & a_{KK} \\
\end{array}
\right]
*
\left[
\begin{array}{c}
x_1 \\
x_2 \\
\vdots \\
x_K \\
\end{array}
\right]
=
\left[
\begin{array}{c}
b_1 \\
b_2 \\
\vdots \\
b_K \\
\end{array}
\right]
\end{equation}
\begin{verbatim}
for(int i=1;i<40;i++)
  printf("`Hello world"');
\end{verbatim}
\begin{algorithmic}
\FOR{i=0,1,$\ldots$,40}
  \item{Wy�wietl napis "Hello world"'}
\ENDFOR
\end{algorithmic}
\begin{equation}
\label{eq:lim}
\lim_{n\rightarrow\infty}\sum_{k-1}^{N}{\frac{1}{k^2}=\frac{\pi^2}{6}}
\end{equation}
\label{eq:lim}
\begin{equation}
\label{eq:16}
[x]_A=\left\{y\in{U}:a(x)=a(y),\forall{a}{\in}{A}\right\}, where~the~central~object~x\in U
\end{equation}
\begin{equation}
\label{eq:20}
cos\left(2\theta)=cos^{2}\theta-sin^{2}\theta
\end{equation}
\begin{equation}
\label{eq:23}
\end{equation}
\end{document}